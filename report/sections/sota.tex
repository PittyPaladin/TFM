\documentclass[../main.tex]{subfiles}
\graphicspath{{./images/}}
%%%%%%%%%%%%%%%%%%%%%%%%%%%%%%%%%%%%%%%%%%%%%%%%%%%%%%%%%%%%%%%%%

\begin{document}



\section{State of the Art}
Object detection is the technology related to detecting and classifying instances of objects of interest on 2D images or 3D point clouds. It has always been one of the main goals of the Computer Vision community since the creation of the discipline. In one way or another, current approaches focus on special features of the objects of interest, whatever those may be, and try to look for those in the input image or cloud. These approaches are very varied and have grown in complexity over the years, specially since the rise of popularity of neural networks and more specifically convolutional neural networks (the surge of neural networks came first to 2D images, and more recently for point cloud data too). So much so that recent year's approaches seem to only rely on them. Being learning methods on the rise in popularity having annotated images has become a must if a project is to be successful, and the more there is the better it may potentially perform. In that direction many datasets have been created to feed the demand for annotated data, which is a good thing. Still, non-neural methods that require less or null volume of data are an option, and a very cost effective one might add, provided the objects to recognize do not vary too much and generalization is not required. 

Applications for object detection are very varied, although most revolve around human detection, car detection, or common object detection. In recent years focus has also moved to object detection in supermarkets and groceries stores, with the aim of automating storage management and reposition. With it, open datasets have also surged in that area.


\end{document}