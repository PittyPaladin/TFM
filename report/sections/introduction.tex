\documentclass[../main.tex]{subfiles}
\graphicspath{{./images/}}

\begin{document}

The aim of this project is to test different algorithms with the purpose of detecting objects in dense environments. Note that by detecting I mean discerning if a certain object is present in an environment, and if it is, find its position. The type of dense environment that will be used for this project will be a simulation of a supermarket shelf, where objects are close together in relatively confined spaces. I will build the testbed with fake food and house furniture to that effect. The environment will be captured with a depth camera, able to output both a 3D point cloud and a 2D image. Hence, the approach is twofold: computer vision algorithms for the 2D image and point cloud-processing related algorithms for the point cloud data.
Even though this may change later, the idea is to use the OpenCV library for the computer vision set of algorithms. Preferably, the Python wrappings will be used for faster testing. For the point cloud processing, the library that will be used is PCL (Point Cloud Library, created by Radu B. Rusu), in C++. The depth camera that will be used is yet to be decided, but chances are that I will use an Intel RealSense D435. Depending on the time and results obtained as the project evolves (and if the need arises) convolutional neural networks may be implemented. Were that to be the case, the library that I would use is TensorFlow. The reason behind these decisions is the fact that I already have experience using both the libraries and the camera. 
 
\end{document}