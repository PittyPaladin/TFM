\documentclass[../main.tex]{subfiles}
\graphicspath{{./images/}}
%%%%%%%%%%%%%%%%%%%%%%%%%%%%%%%%%%%%%%%%%%%%%%%%%%%%%%%%%%%%%%%%%

\begin{document}

\section{Abstract}
This work presents a dual approach to object recognition: 2D computer vision on images on the one hand, and 3D point cloud processing on the other. The recognition of objects has been applied to a synthetic ``supermarket'' shelf (called testbench) with household items, carefully chosen so that they have simple geometry and texture information for the algorithms to use. Specifically, the objects to detect are Pepsi and Coca-Cola cans, Milk bricks and a detergent bottle. After tackling both approaches a combined 2D and 3D approach is performed.

Regarding the 2D object detection part this work takes a very pragmatic approach. First it tackles the issue of the tooling behind creating and curating image datasets, which is an overlooked aspect in many works. It does so by creating a Graphical User Interface with advanced functionality. Then, it starts off the with the simplest detection methods, like Template Matching, all the way to Convolutional Neural Networks, assessing the results of each and their tradeoffs. Regarding the 3D object recognition many vital aspects of point cloud processing are treated first, mainly filtering and compositions of clouds. Then, object recognition itself is addressed, and its limitations are explained. Finally, the 2D and 3D approaches are combined in a modular manner that is quite uncommon in the literature, so as to leverage the strengths and weaknesses of each method.



\end{document}
